\section{Installation instructions}\label{installation-instructions}

To install Numbat, follow these simple instructions.

\subsection{Install MOOSE}\label{install-moose}

Numbat is based on the MOOSE framework, so the first step is to install
MOOSE. For detailed installation instructions depending on your
hardware, see
\href{http://www.mooseframework.com}{www.mooseframework.com}.

\subsection{Clone Numbat}\label{clone-numbat}

The next step is to clone the Numbat repository to your local machine.

In the following, it is assumed that MOOSE was installed to the
directory \emph{\textasciitilde{}/projects.} If MOOSE was installed to a
different directory, the following instructions must be modified
accordingly.

To clone Numbat, use the following commands

\begin{verbatim}
    cd ~/projects
    git clone https://github.com/cpgr/numbat.git
    cd numbat
    git checkout master
\end{verbatim}

\subsection{Compile Numbat}\label{compile-numbat}

Next, compile Numbat using

\begin{verbatim}
    make -jn
\end{verbatim}

where \emph{n} is the number of processing cores on the computer. If
everything has gone well, Numbat should compile without error, producing
a binary named \emph{numbat-opt}.

\subsection{Test Numbat}\label{test-numbat}

Finally, to test that the installation worked, the test suite can be run
using

\begin{verbatim}
    ./run_tests -jn
\end{verbatim}

where \emph{n} is the number of processing cores on the computer. At
this stage, all of the Numbat tests should have run successfully, and
you are ready to run more complicated simulations, see the
\href{/example2D.md}{2D examples} and \href{/example3D.md}{3D examples}
for more details.
