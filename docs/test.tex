\documentclass[12pt]{article}
\renewcommand{\familydefault}{\sfdefault}
\usepackage[paper=letterpaper, top=1in, bottom=1in, left=1in, right=1in]{geometry}
\usepackage{graphicx}
\usepackage[parfill]{parskip}

\usepackage{hyperref}
\hypersetup{colorlinks=true, linkcolor=blue, citecolor=blue, filecolor=blue, urlcolor=blue}

\usepackage{xparse}
\usepackage{tabularx}
\usepackage[table]{xcolor}
\definecolor{code-background}{HTML}{ECF0F1}
\definecolor{info-title}{HTML}{528452}
\definecolor{info}{HTML}{82E0AA}
\definecolor{note-title}{HTML}{3A7296}
\definecolor{note}{HTML}{85C1E9}
\definecolor{important-title}{HTML}{B100B0}
\definecolor{important}{HTML}{FF00FF}
\definecolor{warning-title}{HTML}{968B2B}
\definecolor{warning}{HTML}{FFEC46}
\definecolor{danger-title}{HTML}{B14D00}
\definecolor{danger}{HTML}{F75E1D}
\definecolor{error-title}{HTML}{940000}
\definecolor{error}{HTML}{FFB4B4}

\DeclareDocumentCommand{\admonition}{O{warning-title}O{warning}mm}
{
  \rowcolors{1}{#1}{#2}
  \renewcommand{\arraystretch}{1.5}
  \begin{tabularx}{\textwidth}{X}
    \textcolor[rgb]{1,1,1}{\textbf{#3}} \\ #4
  \end{tabularx}
  \rowcolors{1}{white}{white}
}

\usepackage{listings}
\usepackage{caption}
%\DeclareCaptionFormat{listing}{#1#2#3}
%\captionsetup[lstlisting]{format=listing, singlelinecheck=false, margin=0pt, ont={sf}}
%\definecolor{code-background}{HTML}{ECF0F1}
\lstset{language=sh, basicstyle=\footnotesize\rmfamily, breaklines=true, backgroundcolor=\color{code-background}}


\begin{document}


\tableofcontents\newpage

-   [Installation](../getting\_started/index.html) -   [Governing equations](../governing\_equations/index.html) -   [Available objects](../systems/index.html) -   [Input file syntax](../input\_file\_syntax/index.html)
-   [2D](index.html)
-   [3D](../example3D/index.html) -   [User manual](../download/index.html)[Numbat](../index.html)\{.page-title\}
-   [Getting Started*arrow\textbackslash~\_drop\textbackslash~\_down*](#!)\{.dropdown-button\}
-   [Theory*arrow\textbackslash~\_drop\textbackslash~\_down*](#!)\{.dropdown-button\}
-   [Systems*arrow\textbackslash~\_drop\textbackslash~\_down*](#!)\{.dropdown-button\}
-   [Examples*arrow\textbackslash~\_drop\textbackslash~\_down*](#!)\{.dropdown-button\}
-   [Manual*arrow\textbackslash~\_drop\textbackslash~\_down*](#!)\{.dropdown-button\}
-       *search* *close*
    -   [![](../media/github-mark-light.png)\{.github-mark\}GitHub](https://github.com/cpgr/numbat)\{.github-mark-label\} [example2D](index.html)\{.breadcrumb\}2D examples \{#2d-examples\}
===========

Complete input files for 2D modules using the dimensional and
dimensionless streamfunction formulations are provided, for both
isotropic and anisotropic porous media. These examples are provided in
the Numbat *examples* folder.Isotropic models
----------------

The first 2D examples are for an isotropic porous medium (
).
### Input file

The complete input file for this problem is``` \{style="overflow-y:scroll;max-height:350px"\}
copy# 2D density-driven convective mixing. Instability is seeded by small perturbation
# to porosity. Takes about 5 minutes to run using a single processor.

[Mesh]
  type = GeneratedMesh
  dim = 2
  ymax = 1.5
  nx = 100
  ny = 50
[]

[MeshModifiers]
  [./bias]
    type = NumbatBiasedMesh
    refined\_edge = top
    refined\_resolution = 0.001
    num\_elems = 50
  [../]
[]

[Variables]
  [./concentration]
    initial\_condition = 0
    scaling = 1e6
  [../]
  [./pressure]
    initial\_condition = 1e6
  [../]
[]

[AuxVariables]
  [./noise]
    family = MONOMIAL
    order = CONSTANT
  [../]
[]

[ICs]
  [./noise]
    type = RandomIC
    variable = noise
    max = 0.003
    min = -0.003
  [../]
[]

[Materials]
  [./porosity]
    type = NumbatPorosity
    porosity = 0.3
    noise = noise
  [../]
  [./permeability]
    type = NumbatPermeability
    permeability = '1e-11 0 0 0 1e-11 0 0 0 1e-11'
  [../]
  [./diffusivity]
    type = NumbatDiffusivity
    diffusivity = 2e-9
  [../]
  [./density]
    type = NumbatDensity
    concentration = concentration
    zero\_density = 995
    delta\_density = 10.5
    saturated\_concentration = 0.049306
  [../]
  [./viscosity]
    type = NumbatViscosity
    viscosity = 6e-4
  [../]
[]

[Kernels]
  [./time]
    type = NumbatTimeDerivative
    variable = concentration
  [../]
  [./diffusion]
    type = NumbatDiffusion
    variable = concentration
  [../]
  [./convection]
    type = NumbatConvection
    variable = concentration
    pressure = pressure
  [../]
  [./darcy]
    type = NumbatDarcy
    variable = pressure
    concentration = concentration
  [../]
[]

[BCs]
  [./conctop]
    type = PresetBC
    variable = concentration
    boundary = top
    value = 0.049306
  [../]
  [./Periodic]
    [./x]
      variable = concentration
      auto\_direction = x
    [../]
  [../]
[]

[Preconditioning]
  [./smp]
    type = SMP
    full = true
  [../]
[]

[Executioner]
  type = Transient
  l\_max\_its = 200
  end\_time = 3e5
  solve\_type = NEWTON
  petsc\_options = -ksp\_snes\_ew
  petsc\_options\_iname = '-pc\_type -sub\_pc\_type -ksp\_atol'
  petsc\_options\_value = 'bjacobi ilu 1e-12'
  nl\_abs\_tol = 1e-10
  nl\_max\_its = 25
  dtmax = 2e3
  [./TimeStepper]
    type = IterationAdaptiveDT
    dt = 1
  [../]
[]

[Postprocessors]
  [./boundaryfluxint]
    type = NumbatSideFlux
    variable = concentration
    boundary = top
  [../]
  [./mass]
    type = NumbatTotalMass
    variable = concentration
  [../]
[]

[Outputs]
  [./console]
    type = Console
    perf\_log = true
    output\_nonlinear = true
  [../]
  [./exodus]
    type = Exodus
    file\_base = 2Dddc
    execute\_on = 'INITIAL TIMESTEP\_END'
  [../]
  [./csvoutput]
    type = CSV
    file\_base = 2Dddc
    execute\_on = 'INITIAL TIMESTEP\_END'
  [../]
[]

wzxhzdk:0[(2Dddc.i)](https://github.com/cpgr/numbat/blob/master/examples/2D/anisotropic/2Dddc.i)\{.moose-listing-link
.tooltipped\}Note that the permeability anisotropy is introduced in the kernels using
the *gamma* and *anisotropic\textbackslash~\_tensor* input parameters.

### Running the example \{#running-the-example\_1\}

This example can be run on the commandline using

        numbat-opt -i 2Dddc\_anisotropic.i

Alternatively, this file can be run using the *Peacock* gui provided by
MOOSE using

        peacock -i 2Dddc\_anisotropic.i

in the directory where the input file *2Dddc\textbackslash~\_anisotropic.i* resides.

### Results \{#results\_1\}

This 2D example should take only a few minutes to run to completion,
producing a concentration profile similar to that presented in [Figure
3](#Figure)\{.moose-float-reference\}, where several downwelling plumes of
high concentration can be observed after 5000 s:![](../media/2Danisotropic.png)\{.materialboxed width="100\%"\}Figure 1: 2D concentration profile for (t = 5000 s)In comparison to the isotropic example (with
) presented in [Figure 1](#Figure)\{.moose-float-reference\}, we note that
the concentration profile in the anisotropic example has only reached a
similar depth after 5000 s (compared to 3528 s). The effect of the
reduced vertical permeability in the anisotropic example slows the
convective transport.
This observation can be quantified by comparing the flux per unit width
over the top boundary of both examples, see [Figure
4](#Figure)\{.moose-float-reference\}.![](../media/2Dfluxcomp.png)\{.materialboxed width="100\%"\}Figure 2: Comparison of the 2D flux across the
top boundaryThe inclusion of permeability anisotropy delays the onset of convection
in comparison to the isotropic example, from a time of approximately
2000 seconds in the isotropic example to approximately 3500 seconds in
the anisotropic example.-   [Isotropic models](#isotropic-models)
-   [Anisotropic models](#anisotropic-models)Site Generated with [MOOSE](https://www.mooseframework.org)

[Edit
Markdown](https://github.com/cpgr/numbat/edit/devel/docs/content/example2D.md)\{.moose-edit-markdown\}
\end{document}